\documentclass[slovak]{article}

% Set language
\usepackage[slovak]{babel}
\selectlanguage{slovak}
\usepackage[utf8]{inputenc}

% Change font
\usepackage[T1]{fontenc}
\usepackage{tgheros}
\renewcommand\familydefault{\sfdefault}

% Enable hyperlinks
\usepackage{hyperref}

% Set paper dimensions
\usepackage[a4paper, total={5in, 10in}]{geometry}

% Enable quotations
\usepackage{dirtytalk}

\begin{document}

\section*{Projects}\label{projects}

\textbf{Project title:} Arduino UPS\\
\textbf{Date:} Jul 20, 2017\\
\textbf{Description:} An UPS made of old UPS and Arduino.\\
\textbf{Operating system:} Embedded (Arduino)\\
\textbf{Language:} C\\
\textbf{IDE:} Arduino IDE\\
\textbf{Hardware:} Arduino UNO, LCD KeyPad Shield\\
\textbf{Link:} \href{https://github.com/kyberdrb/Arduino\_UPS}{https://github.com/kyberdrb/Arduino\_UPS}

\begin{center}\rule{3in}{0.4pt}\end{center}

\noindent
\textbf{Project title:} Pong\\
\textbf{Date:} Feb 7, 2017\\
\textbf{Description:} Network multiplayer game \say{Pong}.\\
\textbf{Operating system:} Linux (Arch Linux)\\
\textbf{Language:} C\\
\textbf{Technologies:} SDL, sockets\\
\textbf{IDE:} CodeBlocks\\
\textbf{Link:} \href{https://github.com/kyberdrb/semestralka\_vstavane\_systemy}{https://github.com/kyberdrb/semestralka\_vstavane\_systemy}

\begin{center}\rule{3in}{0.4pt}\end{center}

\noindent
\textbf{Project title:} Keylogger\\
\textbf{Date:} Aug 20, 2017\\
\textbf{Description:} Capture input from a keyboard into a text file.\\
\textbf{Operating system:} Windows\\
\textbf{Language:} C\\
\textbf{IDE:} CodeBlocks\\
\textbf{Link:} \href{https://github.com/kyberdrb/windows\_keylogger}{https://github.com/kyberdrb/windows\_keylogger}

\begin{center}\rule{3in}{0.4pt}\end{center}

\noindent
\textbf{Project title:} Párovačka\\
\textbf{Date:} Qt: 7/2017-8/2017; Python: 7/2018-8/2018\\
\textbf{Description:} A programm for finding an person from a group of people to get a gift. First version of the application was written in C++ and Qt for GUI. Then it was rewritten into Python and Tk for GUI.\\
\textbf{Operating system:} Cross-Platform (Windows/Linux)\\
\textbf{Language:} C++, Python\\
\textbf{Technologies:} Qt, Tkinter\\
\textbf{IDE:} Qt Creator, PyCharm, Visual Studio Code\\
\textbf{Links:}\\
\href{https://github.com/kyberdrb/Parovacka\_Qt}{https://github.com/kyberdrb/Parovacka\_Qt}\\
\href{https://github.com/kyberdrb/Parovacka\_Python}{https://github.com/kyberdrb/Parovacka\_Python}

\begin{center}\rule{3in}{0.4pt}\end{center}

\noindent
\textbf{Project title:} Android application for position tracking of a mobile device\\
\textbf{Date:} Feb 7, 2017\\
\textbf{Description:} Application for gathering information from GPS, Wi-Fi and Bluetooth and its sending to the server.\\
\textbf{Cooperated by:} Departement of Informatics, FRI ŽU\\
\textbf{Operating system:} Android (4.0+)\\
\textbf{Language:} Java\\
\textbf{Techologies:} REST API\\
\textbf{IDE:} Android Studio\\
\textbf{Link:} \href{https://github.com/kyberdrb/PedTrack}{https://github.com/kyberdrb/PedTrack}

\begin{center}\rule{3in}{0.4pt}\end{center}

\noindent
\textbf{Project title:} SDN firewall\\
\textbf{Date:} May 20, 2018\\
\textbf{Description:} SDN firewall is a service implemented for a SDN controller POX. Interaction with the module is provided via a Bash script.\\
\textbf{Cooperated by:} Department of Information Networks, FRI ŽU\\
\textbf{Position in team:} Developer\\
\textbf{Operating system:} Linux (Mininet/Ubuntu/Debian)\\
\textbf{Language:} Python, Bash\\
\textbf{IDE:} Visual Studio Code\\
\textbf{Technologies:} Mininet, POX, OpenFlow (1.0)\\
\textbf{Links:}\\
\href{https://github.com/kyberdrb/FRI/tree/master/Ing/4.semester/Integracia\_Sieti/semestralka}{https://github.com/kyberdrb/FRI/tree/master/Ing/4.semester/Integracia\_Sieti/semestralka}\\
\href{https://github.com/kyberdrb/sdnfirewall}{https://github.com/kyberdrb/sdnfirewall}

\end{document}
