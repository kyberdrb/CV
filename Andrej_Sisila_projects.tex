\documentclass[slovak]{article}

% Set paper dimensions
\usepackage[a4paper, total={5in, 10in}]{geometry}

% Set language
\usepackage[slovak]{babel}
\selectlanguage{slovak}
\usepackage[utf8]{inputenc}

% Change font
\usepackage[T1]{fontenc}
\usepackage{tgheros}
\renewcommand\familydefault{\sfdefault}

\begin{document}

\section*{Projects}\label{projects}

\textbf{Project title:} Arduino UPS\\
\textbf{Date:} Jul 20, 2017\\
\textbf{Description:} An UPS made of old UPS and Arduino. Design and implementation of an UPS on Arduino UNO platform.\\
\textbf{Operating system:} Embedded (Arduino)\\
\textbf{Language:} C\\
\textbf{IDE:} Arduino IDE\\
\textbf{Hardware:} Arduino UNO, Display with keypad (LCD KeyPad Shield)\\
\textbf{Link:} https://github.com/kyberdrb/Arduino\_UPS

\begin{center}\rule{3in}{0.4pt}\end{center}

\noindent
\textbf{Project title:} Pong\\
\textbf{Date:} Feb 7, 2017\\
\textbf{Description:} Network multiplayer game ``Pong''. Desing and implementation of a network multiplayer game ``Pong''.\\
\textbf{Operating system:} Linux (Arch Linux)\\
\textbf{Language:} C\\
\textbf{Technologies:} SDL, sockets\\
\textbf{IDE:} CodeBlocks\\
\textbf{Link:} https://github.com/kyberdrb/semestralka\_vstavane\_systemy

\begin{center}\rule{3in}{0.4pt}\end{center}

\noindent
\textbf{Project title:} Keylogger\\
\textbf{Date:} Aug 20, 2017\\
\textbf{Description:} Capture input from a keyboard into a text file. Modification of an existing keylogger project.\\
\textbf{Operating system:} Windows\\
\textbf{Language:} C\\
\textbf{IDE:} CodeBlocks\\
\textbf{Link:} https://github.com/kyberdrb/windows\_keylogger

\begin{center}\rule{3in}{0.4pt}\end{center}

\noindent
\textbf{Project title:} Párovačka\\
\textbf{Date:} Aug 14, 2018\\
\textbf{Description:} A programm for finding an person from a
group of people to get a gift. GUI and algorithm desing and
implementation. The algorithm finds a person to gift without
reciprocity. First version of the application was written in C++ and Qt
for GUI. Then it was rewritten into Python and Tk for
GUI.\\
\textbf{Operating system:} Multi-platform (Windows/Linux)\\
\textbf{Language:} C++ / Python\\
\textbf{Technologies:} Qt, Tk\\
\textbf{IDE:} Qt Creator, PyCharm, VS
Code\\
\textbf{Links:}\\
https://github.com/kyberdrb/Parovacka\_Qt\\
https://github.com/kyberdrb/Parovacka\_Python

\begin{center}\rule{3in}{0.4pt}\end{center}

\noindent
\textbf{Project title:} Android application for position tracking of a mobile device\\
\textbf{Date:} Feb 7, 2017\\
\textbf{Description:} Application for gathering information from GPS, Wi-Fi and Bluetooth and
their sending to the server. Desing and implementation of GUI,
implementation of object-oriented desing, persistent service, file
caching and REST client for sending files with location
information.\\
\textbf{Ordered / cooperated by:} Departement of Informatics, FRI ŽU\\
\textbf{Operating system:} Android (4.0+)\\
\textbf{Language:} Java\\
\textbf{IDE:} Android Studio\\
\textbf{Link:} https://github.com/kyberdrb/PedTrack

\begin{center}\rule{3in}{0.4pt}\end{center}

\noindent
\textbf{Project title:} SDN firewall\\
\textbf{Date:} May 20, 2018\\
\textbf{Short description:} SDN firewall is a service implemented for a SDN controller\\
\textbf{Ordered / cooperated by:} Department of Information Networks FRI ŽU\\
\textbf{Position in team:} Developer\\
\textbf{Processed work part in details:} Desing and implementation of SDN firewall module for POX SDN controller in Python. Interaction with the module via a Bash script.\\
\textbf{Operating system:} Linux (Mininet/Ubuntu/Debian)\\
\textbf{Language:} Python/Bash\\
\textbf{IDE:} VS Code\\
\textbf{Technologies:} Mininet, POX, OpenFlow (1.0)\\
\textbf{Links:}\\
https://github.com/kyberdrb/FRI/tree/master/Ing/4.semester/Integracia\_Sieti/semestralka\\
https://github.com/kyberdrb/sdnfirewall

\end{document}
